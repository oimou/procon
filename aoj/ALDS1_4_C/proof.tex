\documentclass{article}
\usepackage[dvipdfmx]{graphicx}
\usepackage{parskip}
\usepackage{ascmac}
\usepackage{amsmath,amssymb}
\usepackage{cleveref}
    \crefname{proposition}{命題}{命題}
    \crefname{theorem}{定理}{定理}
    \crefname{lemma}{補題}{補題}
    \crefmultiformat{lemma}{補題~#2#1#3}{,~#2#1#3}{, #2#1#3}{,~#2#1#3}
\usepackage{autonum}
\usepackage{amsthm}
    \makeatletter
    \renewenvironment{proof}[1][\proofname]{\par
        \pushQED{\qed}
        \normalfont
        \topsep6\p@\@plus6\p@ \trivlist
        \item[\hskip\labelsep{\bfseries #1}\@addpunct{\bfseries}]\ignorespaces
    }{%
        \popQED\endtrivlist\@endpefalse
    }
    \renewcommand{\proofname}{証明.}
    \makeatother
\newtheorem{proposition}{命題}
\newtheorem{theorem}{定理}
\newtheorem{lemma}{補題}
\newcommand{\myparagraph}[1]{\paragraph{#1}\mbox{}\\}
\newcommand{\combination}[2]{{}_{#1} \mathrm{C}_{#2}}

\begin{document}

\section{オープンアドレス法に関する議論}

\begin{screen}
    \begin{lemma}
        \label{lemma:1}
        $m, t$を正整数とする。
        $m, t$が互いに素であるための必要十分条件は、
        任意の整数$s, g\ (0 \leq s < m,\ 0 \leq g < m)$に対して
        \begin{equation}
            g \equiv s + jt \mod{m}
        \end{equation}
        を満たすような非負整数$j$が存在することである。
    \end{lemma}
\end{screen}

\begin{proof}
    最初に、十分性を示す。
    $(g, s) = (0, 0),\ (g, s) = (1, 0)$を代入すると、
    \begin{alignat}{2}
        0 &\equiv j t  &\mod{m} \\
        1 &\equiv j' t &\mod{m}
    \end{alignat}
    が成り立つから、ある整数$k$を用いて、
    \begin{equation}
        mk + (j' - j)t = 1
    \end{equation}
    が成り立つ。したがって、$m, t$は互いに素である。よって十分性が示された。

    次に、必要性を示す。
    まず$s = 0$のときを示す。

    ある非負整数$j_0, j_1\ (0 \leq j_0 < j_1 < m)$に対して
    \begin{align}
        g \equiv j_0 t \mod{m} \\
        g \equiv j_1 t \mod{m}
    \end{align}
    が成り立つと仮定すると,
    \begin{equation}
        0 \equiv (j_1 - j_0) t \mod{m}
    \end{equation}
    である。
    $t$は$m$と互いに素であるから,両辺を$t$で割ることができ,
    \begin{equation}
        0 \equiv j_1 - j_0 \mod{m}
    \end{equation}
    よって$j_0 = j_1$であるが,これは$j_0 < j_1$に矛盾する。
    したがって,非負整数$j$を$0 \leq j < m$の範囲で動かすとき,
    $jt$を$m$で割った余りはすべて相異なる。

    このことと鳩の巣原理により,
    \begin{equation}
        \label{eq:s0}
        g \equiv jt \mod{m}
    \end{equation}
    を満たす非負整数$j$が$0 \leq j < m$の範囲に唯一つ存在する。

    つぎに$s \neq 0$のときを示す。
    (\ref{eq:s0})の両辺に$s\ (0 \leq s < m)$を加え,さらに$g + s$を$m$で割った余りを$g'$とおくと,
    \begin{equation}
        g' \equiv s + jt \mod{m}
    \end{equation}
    であり,しかも$0 \leq g' < m$であるから,必要性が示された。
\end{proof}

% この逆も示したい

\end{document}
