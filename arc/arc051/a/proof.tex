\documentclass{article}
\usepackage{parskip}
\usepackage{amsmath}
\begin{document}

次のように点をとる:
\begin{eqnarray*}
    &A(x_1, y_1) \\
    &P_1(x_2, y_2) \\
    &P_2(x_3, y_2) \\
    &P_3(x_3, y_3) \\
    &P_4(x_2, y_3)
\end{eqnarray*}
点$A$を中心とする半径$r$の円$C$の円周および内部が赤く塗られ,
4点$P_1, P_2, P_3, P_4$の長方形$D$の周および内部が青く塗られ,
2つの図形の共通部分が紫に塗られる。

2つの図形が一致することはないので,2つの図形の包含関係は次の3通りのいずれかひとつのみが成り立つ。
\begin{quote}
    i) $C \subset D$ \\
    ii) $D \subset C$ \\
    iii) 上記以外
\end{quote}

ここで,次の同値関係が成り立つ。
\begin{eqnarray*}
    &i) \Longleftrightarrow \mbox{赤い部分が存在しない} \\
    &ii) \Longleftrightarrow \mbox{青い部分が存在しない}
\end{eqnarray*}

まず,赤い部分の存在条件i)を考える。
i)が成り立つのは点$A$が長方形の内部にあって$A$と長方形$D$の各辺との距離がいずれも$r$以上のとき,およびそのときのみであるから,
\[
    x_2 + r \leq x_1 \leq x_3 - r \ \mbox{かつ} \
    y_2 + r \leq y_1 \leq y_3 - r
\]
と同値である。

次に,青い部分の存在条件i)を考える。
赤い部分が存在しないならば青い部分が存在するから,
赤い部分が存在するという前提のもとで青い部分の存在条件ii)を考えればよい。

ii)が成り立つのは点$A$と長方形の各頂点との距離の最大値が$r$以下のとき,およびそのときのみであるから,
\[
    \max \{ AP_1, AP_2, AP_3, AP_4 \} \leq r
\]
すなわち,
\[
    \max \{ AP_1^2, AP_2^2, AP_3^2, AP_4^2 \} \leq r^2
\]
と同値である。

\end{document}
