\documentclass{article}
\usepackage{parskip}
\usepackage{amsmath}
\begin{document}

2つの自然数$m, n$に対して,$f(m, n)$を,$m$を$n$で割った余りと定義すると,
除法の定理により$f(m, n)$は$0 \leq f(m, n) < n$の範囲にただひとつ存在する。

数列$\{r_n\}$を次の漸化式によって定める:
\begin{eqnarray*}
    &r_0 = a,\ r_1 = b \hspace{15pt} (\mbox{$a, b$は正整数}) \\
    &r_n = f(r_{n - 2},\ r_{n - 1}) \hspace{15pt} (n \geq 2)
\end{eqnarray*}
ただし,ユークリッドの互除法の基本原理によれば
$\{r_n\}$は狭義単調減少であって$r_{n + 1} = 0$となるような$n$ (このような$n$を$N$とおく)が存在するから,
割り算の余りが定義できない。
したがって,第$N$項より後は定義しないものとする。

数列$\{r_n\}$に対して,ユークリッドの互除法によって
\begin{equation*}
    \mathrm{gcd}(a, b)
    = \mathrm{gcd}(r_0, r_1)
    = \mathrm{gcd}(r_{N - 1}, r_{N})
    = r_{N}
\end{equation*}
が成り立ち,$N$は本問における「再帰の回数」に一致する。

ここで,フィボナッチ数列$\{F_n\}$を次の漸化式によって定める:
\begin{eqnarray*}
    &F_0 = 0,\ F_1 = 1 \\
    &F_n = F_{n - 2} + F_{n - 1} \hspace{15pt} (n \geq 2)
\end{eqnarray*}
3以上の任意の正整数$k$に対して
\begin{equation*}
    f(F_{k + 1},\ F_{k}) = F_{k - 1}
\end{equation*}
が成り立つから,任意の正整数$K$に対して$a = F_{K + 2},\ b = F_{K + 1}$とおくと,
\begin{eqnarray*}
    &r_0 = F_{K + 2} \\
    &r_1 = F_{K + 1} \\
    &r_2 = f(F_{K + 2},\ F_{K + 1}) = F_{K} \\
    &\vdots \\
    &r_K = f(F_{4},\ F_{3}) = F_2 \\
    &r_{K + 1} = f(F_{3},\ F_{2}) = 0
\end{eqnarray*}
となる。
したがって,$K$が指定されたとき,求める正整数の組$(a,\ b)$のひとつは$(F_{K + 2},\ F_{K + 1})$である。

\end{document}
