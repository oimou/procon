\documentclass{article}
\usepackage[dvipdfmx]{graphicx}
\usepackage{parskip}
\usepackage{ascmac}
\usepackage{amsmath,amssymb}
\usepackage{cleveref}
    \crefname{proposition}{命題}{命題}
    \crefname{theorem}{定理}{定理}
    \crefname{lemma}{補題}{補題}
    \crefname{definition}{定義}{定義}
    \crefmultiformat{lemma}{補題~#2#1#3}{,~#2#1#3}{, #2#1#3}{,~#2#1#3}
\usepackage{autonum}
\usepackage{amsthm}
    \makeatletter
    \renewenvironment{proof}[1][\proofname]{\par
        \pushQED{\qed}
        \normalfont
        \topsep6\p@\@plus6\p@ \trivlist
        \item[\hskip\labelsep{\bfseries #1}\@addpunct{\bfseries}]\ignorespaces
    }{%
        \popQED\endtrivlist\@endpefalse
    }
    \renewcommand{\proofname}{証明.}
    \makeatother
\newtheorem{proposition}{命題}
\newtheorem{theorem}{定理}
\newtheorem{lemma}{補題}
\newtheorem{definition}{定義}
\newcommand{\myparagraph}[1]{\paragraph{#1}\mbox{}\\}
\newcommand{\combination}[2]{{}_{#1} \mathrm{C}_{#2}}

\def\vector#1{\mbox{\boldmath $#1$}}

\begin{document}

所与の整数列に題意の操作を施したものを$N$次元列ベクトルとして扱うことにする。
まず,数列$\{\vector{V}_n\}$を次のように定義する:
\begin{equation}
    \vector{V}_0 = \left(
        \begin{array}{c}
            a_1 \\
            a_2 \\
            \vdots \\
            a_N
        \end{array}
    \right),
    \quad
    \vector{V}_{n+1} = f_A(\vector{V}_n) \quad (n \geq 0)
\end{equation}
ただし,$f_A(\vector{V})$は$\vector{V}$の最小の成分うち最も行番号の小さいものに$A$を掛けたベクトルを返す。
我々は$\vector{V}_B$を求めたい。

$A = 1$のときは題意の操作によって値が変化することはないから$\vector{V}_B = \vector{V}_0$である。
以後$A > 1$のときを考える。

$a_1, a_2, \dots , a_N$はすべて正なので,$\vector{V}_0$の任意の成分について底を$A$とする対数をとることができる。
そこで,数列$\{\vector{v}_n\}$を次のように定義する:
\begin{equation}
    \vector{v}_0 = \left(
        \begin{array}{c}
            \log_A a_1 \\
            \log_A a_2 \\
            \vdots \\
            \log_A a_N
        \end{array}
    \right),
    \quad
    \vector{v_{n+1}} = f(\vector{v}_n) \quad (n \geq 0)
\end{equation}
ただし,$f(\vector{v})$は$\vector{v}$の最小の成分のうち
最も行番号の小さいものに1を足したベクトルを返す\footnote{これは$A$進法におけるシフト演算や,$N$次元空間における軸に平行な距離1の移動と解釈することができる}。
したがって,$f(\vector{v})$によって$\vector{v}$の成分の小数部は変化しない。

ここからは数列$\{\vector{v}_n\}$について考える。
正実数$\log_A a_n$の整数部の表記を簡単にするために,$\log_A a_n$の整数部$\left\lfloor \log_A a_n \right\rfloor$を
$g(a_n)$と表すことにする。

\cref{lemma:loop2}より,項番号を増やしていったときに
$\vector{v}_{B_0}$ではじめて成分の整数部がすべて一致するようなある非負整数$B_0$が存在し\footnote{$B_0$回目から"ループ"が始まる},
$\vector{v}_0$の成分のうち整数部が最大のものの整数部を$I$とおくと,
\begin{equation}
    \label{vB0:1}
    \mbox{$\vector{v}_{B_0}$の成分の整数部はすべて$I$}
\end{equation}
である。ここで,
\begin{equation}
    I = \max \{ g(a_1), \dots, g(a_N) \}
\end{equation}
であり,$B_0$を$I$を用いて表すと,
\begin{equation}
    B_0 = \sum_{k = 1}^{N} \left(
        I - g(a_k)
    \right)
\end{equation}
である。

\cref{lemma:loop1}より,
\begin{equation}
    \label{Dk:1}
    \mbox{任意の非負整数$k$に対して
    $\vector{v}_{B_0 + kN}$の成分の整数部はすべて$I + k$}
\end{equation}
が成り立つ。

\myparagraph{[1] $B > B_0$のとき}

$B - B_0$を$N$で割った余り商を$b$,余りを$B_1$とおき\footnote{$b$はループの回数,$B_1$は最後のループが終わってから$B$回目に至るまでの操作回数},
$\vector{v}_0$の成分の整数部をすべて$I$に置き換えたベクトルを$\vector{v}'_0$,
すべての成分が1であるような$N$次元ベクトルを$\vector{u}$とおくと,
\begin{alignat}{1}
    \vector{v}_B &= \vector{v}_{B_0 + bN + B_1} \\
                 &= f^{B_1}( \vector{v}_{B_0 + bN} ) \\
                 &= f^{B_1}( \vector{v}_{B_0} + b\vector{u} ) \quad (\because (\ref{Dk:1})) \\
                 &= f^{B_1}( \vector{v}'_0 + b\vector{u} ) \quad (\because (\ref{vB0:1}))
\end{alignat}
であり,$\vector{v}_B$の各成分について底を$A$とする指数をとると$\vector{V}_B$を得るから,
\begin{equation}
    \vector{V}_B = f_A^{B_1}\left( \left(
        \begin{array}{c}
            a_1 A^{- g(a_1) + I + b} \\
            a_2 A^{- g(a_2) + I + b} \\
            \vdots \\
            a_N A^{- g(a_N) + I + b}
        \end{array}
    \right) \right)
\end{equation}
である。

\myparagraph{[2] $B \leq B_0$のとき}
\begin{alignat}{1}
    \vector{v}_B &= f^{B}(\vector{v}_0)
\end{alignat}
であり,$\vector{v}_B$の各成分について底を$A$とする指数をとると$\vector{V}_B$を得るから,
\begin{equation}
    \vector{V}_B = f_A^{B}\left( \left(
        \begin{array}{c}
            a_1 \\
            a_2 \\
            \vdots \\
            a_N
        \end{array}
    \right) \right)
\end{equation}
である。

\section{補題}

記号の意味は本文中で用いられているものと同一である。

\begin{itembox}[l]{ループが存在すること}
    \begin{lemma}
        \label{lemma:loop1}
        任意の非負整数$k$に対して次が成り立つ:
        \begin{quote}
            ある整数$I_0$に対して$\vector{v}_k$の成分の整数部がすべて$I_0$ならば,
            $\vector{v}_{k + N}$の成分の整数部はすべて$I_0 + 1$である。
        \end{quote}
    \end{lemma}
\end{itembox}

\begin{proof}
    省略
\end{proof}

\begin{itembox}[l]{有限回の操作でループに突入すること}
    \begin{lemma}
        \label{lemma:loop2}
        任意の$a_1, a_2, \dots, a_N$に対して,ある整数$M$が存在し,
        $\vector{v}_M$の成分の整数部はすべて一致する。
    \end{lemma}
\end{itembox}

\begin{proof}
    省略
\end{proof}

\end{document}
