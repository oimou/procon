\documentclass{article}
\usepackage[dvipdfmx]{graphicx}
\usepackage{parskip}
\usepackage{ascmac}
\usepackage{amsmath,amssymb}
\usepackage{cleveref}
    \crefname{proposition}{命題}{命題}
    \crefname{theorem}{定理}{定理}
    \crefname{lemma}{補題}{補題}
    \crefname{definition}{定義}{定義}
    \crefmultiformat{lemma}{補題~#2#1#3}{,~#2#1#3}{, #2#1#3}{,~#2#1#3}
\usepackage{autonum}
\usepackage{amsthm}
    \makeatletter
    \renewenvironment{proof}[1][\proofname]{\par
        \pushQED{\qed}
        \normalfont
        \topsep6\p@\@plus6\p@ \trivlist
        \item[\hskip\labelsep{\bfseries #1}\@addpunct{\bfseries}]\ignorespaces
    }{%
        \popQED\endtrivlist\@endpefalse
    }
    \renewcommand{\proofname}{証明.}
    \makeatother
\newtheorem{proposition}{命題}
\newtheorem{theorem}{定理}
\newtheorem{lemma}{補題}
\newtheorem{definition}{定義}
\newcommand{\myparagraph}[1]{\paragraph{#1}\mbox{}\\}
\newcommand{\combination}[2]{{}_{#1} \mathrm{C}_{#2}}

\def\vector#1{\mbox{\boldmath $#1$}}

\begin{document}

$A = 1$のときは題意の操作によって成分が変化することはないから,以後$A > 1$のときを考える。

所与の整数列に題意の操作を$n$回施したものを$N$次元ベクトルとして扱うことにする。
次のことを定義しておく。
\begin{itembox}[l]{並び方が等しいふたつのベクトル}
    \begin{definition}
        $N$次元ベクトル$\vector{x}, \vector{y}$に対して
        次のことが成り立つとき,$\vector{x} \equiv \vector{y}$と表す:
        \begin{quote}
            任意の自然数$i, j\ (1 \leq i \leq N, 1 \leq j \leq N)$に対して,
            $\vector{x}, \vector{y}$の$i$番目の成分をそれぞれ$x_i, y_i$,
            $\vector{x}, \vector{y}$の$j$番目の成分をそれぞれ$x_j, y_j$とおくとき,
            $x_i, x_j$の間の大小関係と$y_i, y_j$の間の大小関係が一致する。
        \end{quote}
    \end{definition}
\end{itembox}

まず,数列$\{\vector{V}_n\}$を次のように定義する:
\begin{equation}
    \vector{V}_0 = \left(
        \begin{array}{c}
            a_1 \\
            a_2 \\
            \vdots \\
            a_N
        \end{array}
    \right),
    \quad
    \vector{V}_{n+1} = f_A(\vector{V}_n) \quad (n \geq 0)
\end{equation}
ただし,$f_A(\vector{V})$は$\vector{V}$の最小の成分うち最もはやい\footnotetext[0]{ベクトルの成分を順に並べたときに番目の最も小さいもの}ものに$A$を掛けたベクトルを返す。
$a_1, a_2, \dots , a_N$はすべて正なので,$\vector{V}_0$の任意の成分について底を$A$とする対数をとることができる。
そこで,数列$\{\vector{v}_n\}$を次のように定義する:
\begin{equation}
    \vector{v}_0 = \left(
        \begin{array}{c}
            \log_A a_1 \\
            \log_A a_2 \\
            \vdots \\
            \log_A a_N
        \end{array}
    \right),
    \quad
    \vector{v_{n+1}} = f_1(\vector{v}_n) \quad (n \geq 0)
\end{equation}
ただし,$f_1(\vector{v})$は$\vector{v}$の最小の成分のうち最もはやい\footnotemark[0]ものに1を足したベクトルを返す。
したがって,
\begin{equation}
    \mbox{$f_1(\vector{v})$によってベクトルの成分の小数部は変化しない} \label{f1:1}
\end{equation}

数列$\{\vector{v}_n\}$について考える。次のことを証明しておく。

\begin{screen}
    \begin{lemma}
        \label{lemma:dec}
        任意の自然数$i, j$に対して,
        $\vector{v}_{i}$の成分の整数部がすべて一致しており,
        $\vector{v}_{j}$の成分の整数部がすべて一致しているならば,
        $\vector{v}_i \equiv \vector{v}_j$である。
    \end{lemma}
\end{screen}

\begin{proof}
    任意の自然数$k$に対して,
    $\vector{v}_k$の成分の整数部がすべて一致しているならば,
    $\vector{v}_k$の成分間の大小関係は成分の整数部にはよらず,小数部のみによって決まる。
    このことと(\ref{f1:1})から,\cref{lemma:dec}が成り立つ。
\end{proof}

$f_1(\vector{v})$の性質から明らかに,項番号を増やしていったときに
$\vector{v}_{B_0}$ではじめて成分の整数部がすべて一致するようなある非負整数$B_0$が存在する\footnote{$B_0$回目から"ループ"が始まる}。
$\vector{v}_0$の成分のうち整数部が最大のものの整数部を$I$とおくと,
\begin{equation}
    \label{vB0:1}
    \mbox{$\vector{v}_{B_0}$の成分の整数部はすべて$I$}
\end{equation}
である。ここで,
\begin{equation}
    I = \max \{ \left\lfloor \log_A a_1 \right\rfloor, \dots , \left\lfloor \log_A a_1 \right\rfloor \}
\end{equation}
であり,$B_0$を$I$を用いて表すと,
\begin{equation}
    B_0 = \sum_{k = 1}^{N} \left(
        I - \left\lfloor \log_A a_k \right\rfloor
    \right)
\end{equation}
である。

ふたたび$f_1(\vector{v})$の性質から明らかに,
$\vector{v}_{B_0 + N}$の成分の整数部はすべて$I + 1$となる。
同様に,
\begin{equation}
    \label{Dk:1}
    \mbox{任意の非負整数$k$に対して
    $\vector{v}_{B_0 + kN}$の成分の整数部はすべて$I + k$}
\end{equation}
が成り立つ。
したがって,\cref{lemma:dec}により
\begin{equation}
    \vector{v}_{B_0} \equiv \vector{v}_{B_0 + kN} \quad (\mbox{$k$は任意の非負整数})
\end{equation}
である。

\myparagraph{[1] $B > B_0$のとき}

$B - B_0$を$N$で割った余り商を$b$,余りを$B_1$とおき\footnote{$b$は"ループ"の回数,$B_1$は最後のループが終わってから$B$回目に至るまでの操作回数},
$\vector{v}_0$の成分の整数部をすべて$I$に置き換えたベクトルを$\vector{v}'_0$とおくと,
\begin{alignat}{1}
    \vector{v}_B &= \vector{v}_{B_0 + bN + B_1} \\
                 &= f_1^{B_1}( \vector{v}_{B_0 + bN} ) \\
                 &= f_1^{B_1}( \vector{v}_{B_0} + b ) \quad (\because (\ref{Dk:1})) \\
                 &= f_1^{B_1}( \vector{v}'_0 + b ) \quad (\because (\ref{vB0:1}))
\end{alignat}
であり,$\vector{v}_B$の各成分について底を$A$とする指数をとると$\vector{V}_B$を得る。

\myparagraph{[2] $B \leq B_0$のとき}

\begin{alignat}{1}
    \vector{v}_B &= f_1^{B}(\vector{v}_0)
\end{alignat}
であり,$\vector{v}_B$の各成分について底を$A$とする指数をとると$\vector{V}_B$を得る。

\end{document}
