\documentclass{article}
\usepackage[dvipdfmx]{graphicx}
\usepackage{parskip}
\usepackage{ascmac}
\usepackage{amsmath,amssymb}
\usepackage{cleveref}
    \crefname{proposition}{命題}{命題}
    \crefname{theorem}{定理}{定理}
    \crefname{lemma}{補題}{補題}
    \crefmultiformat{lemma}{補題~#2#1#3}{,~#2#1#3}{, #2#1#3}{,~#2#1#3}
\usepackage{autonum}
\usepackage{amsthm}
    \makeatletter
    \renewenvironment{proof}[1][\proofname]{\par
        \pushQED{\qed}
        \normalfont
        \topsep6\p@\@plus6\p@ \trivlist
        \item[\hskip\labelsep{\bfseries #1}\@addpunct{\bfseries}]\ignorespaces
    }{%
        \popQED\endtrivlist\@endpefalse
    }
    \renewcommand{\proofname}{証明.}
    \makeatother
\newtheorem{proposition}{命題}
\newtheorem{theorem}{定理}
\newtheorem{lemma}{補題}
\newcommand{\myparagraph}[1]{\paragraph{#1}\mbox{}\\}
\newcommand{\combination}[2]{{}_{#1} \mathrm{C}_{#2}}

\begin{document}

$A = 1$のときは題意の操作によって順番が変わることがないから,以下$A > 1$のときを考える。
所与の整数列から2項$a_i, a_j\ (i \neq j)$を任意にとる。
題意の操作を$B$回繰り返したとき,$a_i, a_j$の$A$倍された回数をそれぞれ$m, n$とおくと,
\begin{alignat}{1}
    & A^m a_i \leq A^n a_j \\
    \Longleftrightarrow\ & m \log A + \log a_i \leq n \log A + \log a_j \\
    \Longleftrightarrow\ & \left( \dfrac{\log a_j}{\log A} - \dfrac{\log a_i}{\log A} \right) + (n - m) \geq 0 \label{ineq:1}
\end{alignat}
である。
したがって,項ごとにその$A$倍された回数を記録しておき,操作ごとに(\ref{ineq:1})を用いてソートしてゆけばよい。

\end{document}
