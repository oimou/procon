\documentclass{article}
\usepackage[dvipdfmx]{graphicx}
\usepackage{parskip}
\usepackage{ascmac}
\usepackage{amsmath,amssymb}
\usepackage{cleveref}
    \crefname{proposition}{命題}{命題}
    \crefname{theorem}{定理}{定理}
    \crefname{lemma}{補題}{補題}
    \crefmultiformat{lemma}{補題~#2#1#3}{,~#2#1#3}{, #2#1#3}{,~#2#1#3}
\usepackage{autonum}
\usepackage{amsthm}
    \makeatletter
    \renewenvironment{proof}[1][\proofname]{\par
        \pushQED{\qed}
        \normalfont
        \topsep6\p@\@plus6\p@ \trivlist
        \item[\hskip\labelsep{\bfseries #1}\@addpunct{\bfseries}]\ignorespaces
    }{%
        \popQED\endtrivlist\@endpefalse
    }
    \renewcommand{\proofname}{証明.}
    \makeatother
\newtheorem{proposition}{命題}
\newtheorem{theorem}{定理}
\newtheorem{lemma}{補題}
\newcommand{\myparagraph}[1]{\paragraph{#1}\mbox{}\\}
\newcommand{\combination}[2]{{}_{#1} \mathrm{C}_{#2}}

\begin{document}

\section{本問に関する議論}

素数の集合を$P$,合成数の集合を$C$,「素数っぽい」集合を$Q$,
2, 3, 5のいずれでも割り切れないような2以上の整数の集合を$T$とおく。
題意により,
\begin{equation}
    \begin{cases}
        N \in P \Longrightarrow N \in Q \\
        N \in C\ \mbox{かつ}\ N \in T \Longrightarrow N \in Q
    \end{cases}
\end{equation}
であるから,
\begin{alignat}{1}
    N \in T &\Longrightarrow (N \in P\ \mbox{または}\ N \in C)\ \mbox{かつ}\ N \in T \\
    &\Longrightarrow N \in Q
\end{alignat}
である。

\section{エラトステネスの篩に関する議論}

整数$n$に対して,$n$の倍数のうち$n$と異なるものを$n$の真の倍数と呼ぶことにする。

2以上のすべての整数の集合を$A_0$とおく。
$k$を4以上の整数として,$A_0$の要素のうちで次を満たすものすべての集合を$A_k$とおく:
\begin{quote}
    2以上$\lfloor \sqrt{k} \rfloor$以下のいかなる整数$j$に対してもその真の倍数でない。
\end{quote}

整数$N$に関する条件
\begin{equation}
    x \in A_N\ \mbox{かつ}\ 2 \leq x \leq N
    \Longleftrightarrow
    \mbox{$x$は2以上$N$以下の素数}
\end{equation}
を$P(N)$とおく。
4以上のすべての整数$N$に対して$P(N)$が成り立つことを数学的帰納法によって示す。

\myparagraph{[1] $N = 4$のとき}

$A_4 = \{2, 3, 5, \ldots \}$なので,$P(4)$が成り立つ。

\myparagraph{[2] $N \geq 4$のとき}

ある$N\ (\geq 4)$に対して$P(N)$の成立を仮定する。
$A_{N + 1}$の$N$以下の要素すべての集合は,$A_N$の$N$以下の要素すべての集合に一致するから,
$P(N + 1)$の成立を示すには,
\begin{gather}
    \mbox{$N + 1$が素数である} \\
    \Longleftrightarrow \\
    N + 1 \in A_{N + 1}
\end{gather}
を示せば十分であるが,
\begin{gather}
    \mbox{$N + 1$が素数である} \\
    \Longleftrightarrow \\
    \mbox{$N + 1$が2以上$\lfloor \sqrt{N + 1} \rfloor$以下のいかなる整数でも割り切れない}
\end{gather}
なので,$P(N + 1)$が成り立つ。

以上より,4以上のすべての整数$N$に対して$P(N)$が成り立つ。

\end{document}
