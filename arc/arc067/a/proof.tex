\documentclass{article}
\usepackage{parskip}
\usepackage{amsmath}
\begin{document}

一般に,正整数$n$の素因数分解が素数列$\{p_k\}$と非負整数列$\{\alpha_k\}$を用いて
$$n = p_1^{\alpha_1} p_2^{\alpha_2} \cdots p_k^{\alpha_k}$$
と表されるとき,$n$の正の約数の個数は
$$(\alpha_1 + 1) (\alpha_2 + 1) \cdots (\alpha_k + 1)$$
である。

$N!$の素因数分解に含まれる素因数$p_i$の個数$\alpha_i$は,
$N$以下の正整数の素因数分解に含まれる$p_i$の個数の総和であるから,
整数$m\ (1 \leq m \leq N)$の素因数分解に含まれる素因数$p_i$の個数を
$\alpha_{m, i}$とおくと,$N!$の正の約数の個数は,
\begin{equation}
    \prod_{i}^\infty ( 1 + \sum_{j = 1}^N \alpha_{j, i} ) \nonumber
\end{equation}
である。

\end{document}
