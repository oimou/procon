\documentclass{article}
\usepackage{parskip}
\usepackage{amsmath}
\begin{document}

一般に,正整数$n$の素因数分解が素数列$\{p_k\}$と非負整数列$\{\alpha_k\}$を用いて
$$n = p_1^{\alpha_1} p_2^{\alpha_2} \cdots p_k^{\alpha_k}$$
と表されるとき,$n$の正の約数の個数は
$$(\alpha_1 + 1) (\alpha_2 + 1) \cdots (\alpha_k + 1)$$
である。

ルジャンドルの定理により,
$N!$の素因数分解に含まれる素因数$p$の個数$\alpha_i$は,
\begin{equation}
    \sum_{j = 1}^\infty \left\lfloor \dfrac{N}{p^j} \right\rfloor \nonumber
\end{equation}
であるから,$N!$の正の約数の個数は,
\begin{equation}
    \prod_{i}^\infty ( 1 + \sum_{j = 1}^\infty \left\lfloor \dfrac{N}{p_i^j} \right\rfloor ) \nonumber
\end{equation}
である。

\end{document}
