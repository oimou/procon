\documentclass{article}
\usepackage{parskip}
\usepackage{amsmath}
\begin{document}

$N$を3以上の整数とすると,$N$は素数あるいは合成数のいずれかである。
$N$が合成数ならば,2つの正整数$n_1, n_2\ (2 \leq n_1 \leq n_2 \leq N - 1)$を用いて
$N = n_1 n_2$と表すことができ,
$\lceil \sqrt{N} \rceil \geq \sqrt{N}$より$\lceil \sqrt{N} \rceil^2 \geq N$
であることに注意すれば,$n_1 \leq \lceil \sqrt{N} \rceil$である。
実際,$n_1 > \lceil \sqrt{N} \rceil$と仮定すると,
\begin{eqnarray*}
    n_1 n_2 &>&    \lceil \sqrt{N} \rceil n_2 \\
            &\geq& \lceil \sqrt{N} \rceil n_1 \\
            &>&    \lceil \sqrt{N} \rceil^2 \\
            &\geq& N
\end{eqnarray*}
となり矛盾する。

したがって,次が成り立つ:
\[
    \mbox{$N$が合成数} \Longrightarrow \mbox{$N$は$2 \leq n \leq \lceil \sqrt{N} \rceil$なる約数$n$を持つ}
\]
また,素数の定義から明らかに次が成り立つ:
\[
    \mbox{$N$が素数} \Longrightarrow \mbox{$N$は$2 \leq n \leq \lceil \sqrt{N} \rceil$なる約数$n$を持たない}
\]
したがって,次が成り立つ:
\[
    \mbox{$N$が素数} \Longleftrightarrow \mbox{$N$は$2 \leq n \leq \lceil \sqrt{N} \rceil$なる約数$n$を持たない}
\]

\end{document}
