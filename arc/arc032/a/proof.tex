\documentclass{article}
\usepackage{parskip}
\usepackage{amsmath}
\begin{document}

$n = 1$のとき,総和は1であり,これは素数でない。

$n = 2$のとき,総和は3であり,これは素数である。

$n >= 3$のとき,1から$n$までの整数の総和は
$$\frac{1}{2} n (n + 1)$$
であり,$n$と$n + 1$は偶奇が相異なることに注意すれば,
これは1より大きな2つの整数の積である。すなわち素数でない。

\end{document}
