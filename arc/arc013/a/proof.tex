\documentclass{article}
\usepackage{parskip}
\usepackage{amsmath}
\begin{document}

$x, y, z$軸をそれぞれタテ,ヨコ,高さの向きにとる。

荷物の向きの選び方の個数は,$x, y, z$軸に平行な辺の選び方の個数に等しいから,${}_3 \mathrm{P}_2 = 6$通りである。

このそれぞれに対して,$x, y, z$軸に平行な辺の長さを$E_x, E_y, E_z$とおくと,
$x, y, z$軸方向に並べることができる最大の個数はそれぞれ,
\[
    \left\lfloor \cfrac{N}{E_x} \right\rfloor,
    \left\lfloor \cfrac{M}{E_y} \right\rfloor,
    \left\lfloor \cfrac{L}{E_z} \right\rfloor
\]
であるから,梱包できる荷物の個数は,
\begin{equation}
    \label{packages}
    \left\lfloor \cfrac{N}{E_x} \right\rfloor
    \left\lfloor \cfrac{M}{E_y} \right\rfloor
    \left\lfloor \cfrac{L}{E_z} \right\rfloor
\end{equation}
である。
したがって,求める値は6通りのうちで最大の(\ref{packages})の値である。

\end{document}
