\documentclass{article}
\usepackage{parskip}
\usepackage{amsmath}
\begin{document}

問題文に与えられた「よい文字列」がみたす条件を箇条書きの上から順にi), ii), iii)とおく。
$l = \mathrm{lcm}(N, M)$ とおいて,
「よい文字列」が存在するならば $|X| = l$ のときに存在することを示す。

よい文字列が存在するとき,これを$X'$とおくと,
$|X'|$は$N$でも$M$でも割り切れるから,$|X'| = kl$なる非負整数$k$が存在する。

$NM = \mathrm{gcd}(N, M) \times \mathrm{lcm}(N, M)$により,
\begin{eqnarray*}
    \frac{l}{N} = \frac{M}{\mathrm{gcd}(N, M)} \hspace{15pt} (= M' \mbox{とおく}) \\
    \frac{l}{M} = \frac{N}{\mathrm{gcd}(N, M)} \hspace{15pt} (= N' \mbox{とおく})
\end{eqnarray*}
であるから,$M'$と$N'$は互いに素である。
$M', N'$を用いて条件ii), iii)を表すと,
\begin{eqnarray*}
    X'\ \mbox{の}\ 1,\ kM'+1,\ 2kM'+1,\ \ldots ,\ (N-1)kM'+1\ \mbox{番目の連結は}\ S \\
    X'\ \mbox{の}\ 1,\ kN'+1,\ 2kN'+1,\ \ldots ,\ (M-1)kN'+1\ \mbox{番目の連結は}\ T
\end{eqnarray*}
となる。

文字列$S, T$をなす文字の位置,すなわち$1 \leq i \leq N - 1,\ 1 \leq j \leq M - 1$に対する
$(i - 1)kM' + 1,\ (j - 1)kN' + 1$の値をそれぞれ$M', N'$で割った余りがすべて1であることに注意すると,
中国剰余定理により,文字の位置は$M'N'$ごとに唯一つ重なる。

% もう少し厳密に

% このあとがわからない

% 最後に,Xに対してではなくS,Tに対する条件が導かれると思うので,それを実装に利用する

\end{document}
