\documentclass{article}
\usepackage{parskip}
\usepackage{amsmath}
\usepackage{amssymb}
\begin{document}

問題文に与えられた「よい文字列」がみたす条件を上から順にi), ii), iii)とおき,
$l = \mathrm{lcm}(N, M)$ とおく。

まず,次を示す。
\begin{eqnarray}
    \label{prop2}
    &\mbox{「よい文字列」が存在するならば,その長さは$l$以上である} \\
    \label{prop1}
    &\mbox{「よい文字列」が存在するならば,その長さが$l$のときにも存在する}
\end{eqnarray}

よい文字列が存在するとき,これを$Y$とおくと,
条件i)により,正整数$k$を用いて$|Y| = kl$と表せることが必要である。
したがって(\ref{prop2})が成り立つ。

$NM = \mathrm{gcd}(N, M) \times \mathrm{lcm}(N, M)$により,
\begin{eqnarray*}
    \frac{l}{N} = \frac{M}{\mathrm{gcd}(N, M)} \hspace{15pt} (= M' \mbox{とおく}) \\
    \frac{l}{M} = \frac{N}{\mathrm{gcd}(N, M)} \hspace{15pt} (= N' \mbox{とおく})
\end{eqnarray*}
が成り立ち,$M'$と$N'$は互いに素である。
$M', N'$を用いて条件ii), iii)を表すと,
\begin{eqnarray*}
    Y\ \mbox{の}\ 1,\ kM'+1,\ 2kM'+1,\ \ldots ,\ (N-1)kM'+1\ \mbox{番目の連結は}\ S \\
    Y\ \mbox{の}\ 1,\ kN'+1,\ 2kN'+1,\ \ldots ,\ (M-1)kN'+1\ \mbox{番目の連結は}\ T
\end{eqnarray*}
となる。

ここで,文字列$Y$を"圧縮"して,長さ$l$の新たな文字列$X$をつくることを考える。
$Y, X$の$n$文字目の文字をそれぞれ$y_n, x_n$とおくと,
$1 \leq i < N,\ 1 \leq j < M$なるすべての整数$i, j$に対して
\begin{equation}
    \label{compression}
    \begin{cases}
        y_{(i - 1)kM' + 1} = x_{(i - 1)M' + 1} \\
        y_{(j - 1)kN' + 1} = x_{(j - 1)N' + 1}
    \end{cases}
\end{equation}
を満たす文字列$X$が存在するならば$X$もまた「よい文字列」であるが,
\begin{eqnarray}
    (i - 1)kM' + 1 = (j - 1)kN' + 1
        &\Longleftrightarrow& (i - 1)kM' = (j - 1)kN' \nonumber \\
        &\Longleftrightarrow& (i - 1)M' = (j - 1)N' \nonumber \\
        \label{eq1}
        &\Longleftrightarrow& (i - 1)M' + 1 = (j - 1)N' + 1
\end{eqnarray}
により,"圧縮"の前後で文字の一致のしかたは変わらないといえるので,
(\ref{compression})を満たす$X$が存在する。
したがって(\ref{prop1})が成り立つ。

つぎに,$|X| = l$のもとで,$X$が存在するために文字列$S, T$が満たすべき条件を考える。
(\ref{eq1})と,$M', N'$が互いに素であることから,任意の整数$d$を用いて
\begin{eqnarray}
    (\ref{eq1})
    \Longleftrightarrow \begin{cases}
        i = N'd + 1 \\
        j = M'd + 1
    \end{cases}
    \nonumber
\end{eqnarray}
が成り立つ。ただし,$i, j$の範囲に関する条件から,$d$の範囲は
\begin{equation}
    \label{range_d}
    0 \leq d < \mathrm{gcd}(N, M)
\end{equation}
である。
したがって,$S, T$の$n$文字目の文字をそれぞれ$s_n, t_n$とおくと,
$X$が存在するためには,
\begin{equation}
    \mbox{(\ref{range_d})を満たす整数$d$によって定まるすべての$i, j$に対して$s_i = t_j$}
    \nonumber
\end{equation}
が必要十分である。

\end{document}
