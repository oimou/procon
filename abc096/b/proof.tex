\documentclass{article}
\usepackage{parskip}
\usepackage{cases}
\usepackage{bm}
\begin{document}

3つの数字の大小関係を,
\begin{equation}
\label{abc}
A \leq B \leq C
\end{equation}
と仮定しても一般性を失わない。

$n$回の操作を終えた後の3つの数字の合計としてありうる最大の値を$s_n$とおくと,$s_n$は,
\begin{equation}
S = \{\ 2^a A + 2^b B + 2^c C \ |\ a,b,c \mbox{は非負整数}, a + b + c = n\ \}
\end{equation}
によって定まる集合$S$の要素のうち最大のものであり,並び替え不等式より,
\begin{equation}
s_n = A + B + 2^n C
\end{equation}
である。

\end{document}
