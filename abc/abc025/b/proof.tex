\documentclass{article}
\usepackage[dvipdfmx]{graphicx}
\usepackage{parskip}
\usepackage{ascmac}
\usepackage{amsmath,amssymb}
\usepackage{cleveref}
    \crefname{proposition}{命題}{命題}
    \crefname{theorem}{定理}{定理}
    \crefname{lemma}{補題}{補題}
    \crefmultiformat{lemma}{補題~#2#1#3}{,~#2#1#3}{, #2#1#3}{,~#2#1#3}
\usepackage{autonum}
\usepackage{amsthm}
    \makeatletter
    \renewenvironment{proof}[1][\proofname]{\par
        \pushQED{\qed}
        \normalfont
        \topsep6\p@\@plus6\p@ \trivlist
        \item[\hskip\labelsep{\bfseries #1}\@addpunct{\bfseries}]\ignorespaces
    }{%
        \popQED\endtrivlist\@endpefalse
    }
    \renewcommand{\proofname}{証明.}
    \makeatother
\newtheorem{proposition}{命題}
\newtheorem{theorem}{定理}
\newtheorem{lemma}{補題}
\newcommand{\myparagraph}[1]{\paragraph{#1}\mbox{}\\}
\newcommand{\combination}[2]{{}_{#1} \mathrm{C}_{#2}}

\begin{document}

東向きを正、西向きを負と定める。関数$\mathrm{clamp}(x, a, b)$を
\begin{equation}
    \mathrm{clamp}(x, a, b) =
    \begin{cases}
        a & (x < a \mbox{のとき}) \\
        x & (a \leq x \leq b \mbox{のとき}) \\
        b & (x > b \mbox{のとき})
    \end{cases}
\end{equation}
と定義すると、求める値は
\begin{equation}
    \sum_{i = 1}^{N} \left( S_i \times \mathrm{clamp}(d_i, A, B) \right)
\end{equation}
である。ただし、$S_i$の値は文字列$s_i$が East ならば 1,West ならば -1 とする。

\end{document}
