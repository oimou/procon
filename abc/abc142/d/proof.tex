\documentclass{article}
\usepackage{parskip}
\usepackage{amsmath}
\begin{document}

$A$と$B$の公約数の集合を$R$とおき,$A$と$B$の共通の素因数の集合と$\{1\}$との和集合を$S$とおく。
すなわち,たとえば$A = 12, B = 18$のときは,
\[
    R = \{1, 2, 3, 6\}, S = \{1, 2, 3\}
\]
である。

$R$の部分集合でその要素が対ごとに素であるような集合を「問題の集合」と呼ぶことにし,
「問題の集合」のうち要素の個数が最大のものを$R_0$とおくと,
$R_0$の要素の個数が求める値である。

$R, S$に対して次が成り立つ:
\begin{quote}
    性質 i) $S \subset R$ \\
    性質 ii) $R$に含まれる任意の合成数$c$について,$c$の素因数はすべて$S$に含まれる
\end{quote}

性質 i)について,$p \in S$ならば$p$は$A$も$B$も割り切るので$p \in R$,ゆえに$S \subset R$である。

性質 ii)について,$R$に含まれる任意の合成数$c$とその任意の素因数$p$をとると,
$p$が$c$を割り切ることと,$c$が$A$も$B$も割り切ることから,$p$は$A$も$B$も割り切る。
したがって$p$は$A$と$B$の共通の素因数なので,$p \in S$である。
よって性質 ii)が成り立つ。

ここで,$R_0$に合成数$c$が含まれると仮定すると,
$R_0$から$c$を除いて代わりに$c$の素因数を含めた集合を$R_0'$とおけば,
性質 ii)によって$R_0'$もまた「問題の集合」であるが,その要素の個数は$R_0$より多いので,
$R_0$の定義に矛盾する。よって$R_0$は合成数を含まない。

したがって,$R_0$は$R$に含まれる素因数の集合と$\{1\}$との和集合であり,
性質 i)によって$R_0 = S$である。

\end{document}
