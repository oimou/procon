\documentclass{article}
\usepackage{parskip}
\usepackage{amsmath}

\newcommand{\argmin}{\mathop{\rm arg~min}\limits}

\begin{document}

簡単のため,$x_i = p$とおく。

西端を原点とし西から東に進む向きを正として$x$軸をとる。
$x = p$なる点を$P_i$とおき,$x = s_j, x = t_j$なる点をそれぞれ$S_j, T_j$とおく。

同種の寺社のうち最初に通過するもののみを「訪れた」とみなすことにすると,
始点$P_i$を出発して神社1個と寺1個を訪れるような移動の仕方は,
\begin{quote}
    i) $P_i$から進む向き 2通り \\
    ii) 1番目に訪れる寺社の種類の選び方 2通り \\
    iii) 1番目に訪れた寺社から進む向き 2通り
\end{quote}
より,計$2^3 = 8$通りである。

始点に最も近い神社,寺の座標はそれぞれ$p$の関数であることから,次のように表せる:
\begin{quote}
    始点以東で最も近い神社,寺の座標をそれぞれ$f_+(p),\ g_+(p)$ \\
    始点以西で最も近い神社,寺の座標をそれぞれ$f_-(p),\ g_-(p)$ \\
    始点から最も近い神社,寺の座標をそれぞれ$f(p),\ g(p)$
\end{quote}

ただし,始点以東,以西に寺社が存在しない場合は$f_\pm(p) = \pm\infty,\ g_\pm(p) = \pm\infty$(複号同順)とする。

[1] i),iii)で同方向に進むとき
\begin{quote}
    ii)について,始点に最も近い寺社を1番目に訪れるしかないから,
    求める最小の移動距離は,
    \begin{equation}
        \label{expr1}
        \max\{ |f_\pm(p) - p|,\ |g_\pm(p) - p| \} \ (\mbox{複号同順})
    \end{equation}
\end{quote}

[2] i),iii)で逆方向に進むとき
\begin{quote}
ii)について,1番目に訪れる寺社の選び方は2通りあるが,
始点と1番目に訪れる寺社との間に2番目に訪れる寺社がないことに注意すれば,
移動距離は
\begin{eqnarray*}
    2 \times (\mbox{始点と1番目に訪れる寺社との距離}) \\
           + (\mbox{始点と2番目に訪れる寺社との距離})
\end{eqnarray*}
と表せるので,始点から最も近い寺社に1番目に訪れると移動距離が最小となる。
したがって,求める最小の移動距離は,
\begin{equation}
    \label{expr2}
    2 \min\{ |f(p) - p|,\ |g(p) - p| \} + \max\{ |f(p) - p|,\ |g(p) - p| \}
\end{equation}
\end{quote}

以上より,求める最小の移動距離は(\ref{expr1})と(\ref{expr2})のうち最小のものである。

\end{document}
