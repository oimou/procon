\documentclass{article}
\usepackage{parskip}
\begin{document}
3つの数字の大小関係を,
$$A \leq B \leq C$$
と仮定しても一般性を失わない。

パネルの並べ方は次の$6$通り:
\begin{eqnarray}
    &[A][B][+][C] \label{a} \\
    &[A][C][+][B] \label{a_} \\
    &[B][A][+][C] \label{b} \\
    &[B][C][+][A] \label{b_} \\
    &[C][A][+][B] \label{c} \\
    &[C][B][+][A] \label{c_}
\end{eqnarray}
ここで,(\ref{a})と(\ref{a_}),(\ref{b})と(\ref{b_}),(\ref{c})と(\ref{c_})
が表す数式の値がそれぞれ互いに等しいことに注意すれば,ありうる数式の値は次の$3$通り:
\begin{eqnarray}
    &10A + B + C \\
    &A + 10B + C \\
    &A + B + 10C
\end{eqnarray}
並べ替え不等式より,最大値は$$A + B + 10C$$である。
\end{document}
