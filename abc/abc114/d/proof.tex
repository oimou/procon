\documentclass{article}
\usepackage[dvipdfmx]{graphicx}
\usepackage{parskip}
\usepackage{ascmac}
\usepackage{amsmath,amssymb}
\usepackage{cleveref}
    \crefname{proposition}{命題}{命題}
    \crefname{theorem}{定理}{定理}
    \crefname{lemma}{補題}{補題}
    \crefmultiformat{lemma}{補題~#2#1#3}{,~#2#1#3}{, #2#1#3}{,~#2#1#3}
\usepackage{autonum}
\usepackage{amsthm}
    \makeatletter
    \renewenvironment{proof}[1][\proofname]{\par
        \pushQED{\qed}
        \normalfont
        \topsep6\p@\@plus6\p@ \trivlist
        \item[\hskip\labelsep{\bfseries #1}\@addpunct{\bfseries}]\ignorespaces
    }{%
        \popQED\endtrivlist\@endpefalse
    }
    \renewcommand{\proofname}{証明.}
    \makeatother
\newtheorem{proposition}{命題}
\newtheorem{theorem}{定理}
\newtheorem{lemma}{補題}
\newcommand{\myparagraph}[1]{\paragraph{#1}\mbox{}\\}
\newcommand{\combination}[2]{{}_{#1} \mathrm{C}_{#2}}

\begin{document}

一般に、ある正整数$n$の素因数分解が相異なる素数$p_1, p_2, \dots, p_k$を用いて
\begin{equation}
    n = p_1^{\alpha_1} p_2^{\alpha_2} \dots p_k^{\alpha_k}
\end{equation}
と表せるとき、$n$の正の約数の個数は
\begin{equation}
    (\alpha_1 + 1)(\alpha_2 + 1) \dots (\alpha_k + 1)
\end{equation}
である。

$75 = 3 \cdot 5^2$なので、七五数は相異なる素数$a, b, c$を用いて
\begin{equation}
    a^4 b^4 c^2 \quad \mbox{または} \quad a^{14} b^4 \quad \mbox{または} \quad a^{24} b^2
    \quad \mbox{または} \quad a^74
\end{equation}
と表せる数である。

$N!$の素因数のうち、$N!$の素因数分解に2個以上、4個以上、14個以上、24個以上、74個以上含まれるようなものの集合を
それぞれ$G_2, G_4, G_{14}, G_{24}, G_{74}$とおき、
\begin{alignat}{1}
    d_0 &= |G_{24}| \\
    d_1 &= |G_2 \setminus G_{24}| \\
    d_2 &= |G_{14}| \\
    d_3 &= |G_4 \setminus G_{14}| \\
    d_4 &= |G_4| \\
    d_5 &= |G_2 \setminus G_4| \\
    d_6 &= |G_{74}|
\end{alignat}
とおく。

求める七五数の個数は、
\begin{alignat}{1}
    &d_0 (d_0 - 1) + d_1 d_0 \\
    &+ d_2 (d_2 - 1) + d_3 d_2 \\
    &+ d_4 \times \combination{d_4 - 1}{2} + d_5 \times \combination{d_4}{2} \\
    &+ d_6
\end{alignat}
である。

\end{document}
