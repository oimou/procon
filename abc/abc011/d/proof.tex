\documentclass{article}
\usepackage{parskip}
\usepackage{amsmath}
\usepackage{ascmac}
\usepackage{cases}
\begin{document}

\section{証明}

点$(X, Y)$を$P$とおく。
移動の対称性により,$P$は$X \geq 0, Y \geq 0$の範囲にあるとしても一般性を失わない。
以下,この前提のもとで考える。

$P$に到達できる移動方法が存在するためには,$X$も$Y$も$D$で割り切れることが必要である。
このとき,ある非負整数$k, l$が存在し,次が成り立つ。
\begin{eqnarray*}
    X &=& kD \\
    Y &=& lD
\end{eqnarray*}
よって,原点から$P$までのマンハッタン距離は$X + Y = (k + l)D$と表すことができる。
このことと,$N$回で移動できるマンハッタン距離の最大値は$ND$であることから,
\begin{equation*}
    N \geq k + l
\end{equation*}
が必要である。

ここで,$X, Y$軸に平行な移動の回数をそれぞれ$A, B$とおくと,
\begin{eqnarray*}
    &N = A + B \\
    &k \leq A \leq N
\end{eqnarray*}
である。

さらに,$+x,\ -x,\ +y,\ -y$方向の移動の回数をそれぞれ$A_+,\ A_-,\ B_+,\ B_-$とおく。
各軸に平行な移動について "行き過ぎた分は戻らないといけない" ことに注意すれば,
\begin{eqnarray*}
    &A = A_+ + A_- \\
    &B = B_+ + B_- \\
    &k = A_+ - A_- \\
    &l = B_+ - B_-
\end{eqnarray*}
が成り立つ。

$A$を固定すると,他の方向の移動の回数もすべて決まる。
$x, y$軸に平行な移動の仕方は互いに独立であり,それぞれの個数は,
\begin{eqnarray*}
    {}_A \mathrm{C}_{A_+},\ 
    {}_B \mathrm{C}_{B_+}
\end{eqnarray*}
であるから,固定された$A$に対して,点$P$にちょうど到達する確率は,
\begin{equation*}
    \dfrac{{}_A \mathrm{C}_{A_+}}{2^A}
    \dfrac{{}_B \mathrm{C}_{B_+}}{2^B}
\end{equation*}
である。

したがって,$A$の固定を解除すると,求める確率は,
\begin{equation}
    \label{answer}
    \sum_{A}
    \left(\dfrac{{}_A \mathrm{C}_{A_+}}{2^A}
    \dfrac{{}_B \mathrm{C}_{B_+}}{2^B}\right)
\end{equation}
である。
ただし,$A_+ = \dfrac{A + k}{2}$なる整数$A_+$が存在するためには
$A$と$k$の偶奇が一致しなければならないから,
非負整数$t$を用いて
\begin{equation*}
    A = k + 2t
\end{equation*}
を満たすことが必要である。
$t$の範囲は,$A, B$の範囲に注意すれば,
\begin{equation*}
    0 \leq t \leq \dfrac{N - (k + l)}{2}
\end{equation*}
である。



\section{計算手法に関する証明}

一般に,非負整数$n, r\ (n \geq 2,\ 2 \leq r \leq n - 1)$に対して次が成り立つ:
\begin{equation*}
    {}_n \mathrm{C}_r = {}_{n-1} \mathrm{C}_{r-1} + {}_{n-1} \mathrm{C}_{r}
\end{equation*}
このことと二項係数の定義から,
非負整数$s, n, r\ (r \leq n)$に対して$f(n, r) = \dfrac{{}_n \mathrm{C}_r}{s^n}$とおくと,
次の漸化式が成り立つ。
\begin{eqnarray*}
    &f(n, 0) = \dfrac{1}{s^n},\ f(n, n) = \dfrac{1}{s^n} \\
    &f(n, r) = \dfrac{1}{s} \{f(n-1, r-1) + f(n-1, r)\} \\
\end{eqnarray*}
このことを用いて,再帰的に確率の値を計算することができる。

\end{document}
