\documentclass{article}
\usepackage{parskip}
\usepackage{amsmath}
\usepackage{ascmac}
\usepackage{cases}
\begin{document}

\section{証明}

点$(X, Y)$を$P$とおく。
移動の対称性により,$P$は$X \geq 0, Y \geq 0$の範囲にあるとしても一般性を失わない。
以下,この前提のもとで考える。

$P$に到達できる移動の仕方が存在するためには,$X$も$Y$も$D$で割り切れることが必要である。
このとき,ある非負整数$k, l$が存在し,次が成り立つ。
\begin{equation}
    \label{k_l}
    \begin{cases}
        X = kD \\
        Y = lD
    \end{cases}
\end{equation}
よって,原点から$P$までのマンハッタン距離は$X + Y = (k + l)D$と表すことができる。
このことと,$N$回で移動できるマンハッタン距離の最大値は$ND$であることから,
\begin{equation}
    \label{N_k_l}
    N \geq k + l
\end{equation}
が必要である。

ここで,$X, Y$軸に平行な移動の回数をそれぞれ$A, B$とおくと,
\begin{equation*}
    \begin{cases}
        N = A + B \\
        k \leq A \leq N
    \end{cases}
\end{equation*}
である。

さらに,$+x,\ -x,\ +y,\ -y$方向の移動の回数をそれぞれ$A_+,\ A_-,\ B_+,\ B_-$とおく。
各軸に平行な移動について "行き過ぎた分は戻らないといけない" ことにも注意すれば,
\begin{equation*}
    \begin{cases}
        A = A_+ + A_- \\
        B = B_+ + B_- \\
        k = A_+ - A_- \\
        l = B_+ - B_-
    \end{cases}
    \hspace{15pt} \mbox{i.e.} \hspace{15pt}
    \begin{cases}
        A_\pm = \cfrac{A \pm k}{2} \hspace{15pt} \mbox{(複号同順)} \\
        B_\pm = \cfrac{B \pm l}{2} \hspace{15pt} \mbox{(複号同順)}
    \end{cases}
\end{equation*}
が成り立つ。
よって,$A$を固定すると他の方向の移動の回数もすべて決まる。
移動の仕方の個数は次の選び方の個数に等しい:
\begin{quote}
    i) $N$回の移動のうち$x$軸に平行な移動をする$A$回を選ぶ \\
    ii) このそれぞれに対して,$x$軸に平行な$A$回の移動のうち$+x$方向に移動する$A_+$回を選ぶ \\
    iii) このそれぞれに対して,$y$軸に平行な$B$回の移動のうち$+y$方向に移動する$B_+$回を選ぶ
\end{quote}

$N$回の移動の仕方は$4^N$通りあり,これらはの起こり方は同様に確からしいから,求める確率は,
\begin{equation}
    \label{answer}
    \dfrac{1}{4^N}
    \sum_{A}
    \left(
    {}_N \mathrm{C}_{A}
    \times
    {}_A \mathrm{C}_{A_+}
    \times
    {}_B \mathrm{C}_{B_+}
    \right)
    =
    \sum_{A}
    \left(
    \dfrac{{}_N \mathrm{C}_{A}}{2^N}
    \dfrac{{}_A \mathrm{C}_{A_+}}{2^A}
    \dfrac{{}_B \mathrm{C}_{B_+}}{2^B}
    \right)
\end{equation}
である。

つぎに,$A$の範囲を求める。
$A_\pm,\ B_\pm$がいずれも整数であるためには
$A$と$k$, $B$と$l$の偶奇がそれぞれ一致しなければならないから,
非負整数$t, u$を用いて
\begin{equation*}
    \begin{cases}
        A = k + 2t \\
        B = l + 2u
    \end{cases}
\end{equation*}
が成り立つことが必要十分であり,
$N = A + B$を用いて整理すると,$t, u$が
\begin{equation*}
    t - u = \dfrac{1}{2} (N - (k + l))
\end{equation*}
を満たすことが必要十分である。
このような整数$t, u$が存在するためには,
\begin{equation}
    \label{N_k_l_even}
    N - (k + l) \mbox{が偶数}
\end{equation}
が必要である。
また,$A, B$が$k \leq A \leq N,\ l \leq B \leq N$を満たすための$t$の条件は,
\begin{equation}
    \label{t_range}
    0 \leq t \leq \dfrac{N - (k + l)}{2}
\end{equation}
である。

以上より,求める確率は,
(\ref{k_l}), (\ref{N_k_l}), (\ref{N_k_l_even})が成り立つならば,
(\ref{t_range})の範囲のすべての$t$により定まる$A$に関する和(\ref{answer})であり,
そうでない場合は0である。



\section{計算手法に関する証明}

一般に,非負整数$n, r\ (n \geq 2,\ 2 \leq r \leq n - 1)$に対して次が成り立つ:
\begin{equation*}
    {}_n \mathrm{C}_r = {}_{n-1} \mathrm{C}_{r-1} + {}_{n-1} \mathrm{C}_{r}
\end{equation*}
このことと二項係数の定義から,
非負整数$s, n, r\ (r \leq n)$に対して$f(n, r) = \dfrac{{}_n \mathrm{C}_r}{s^n}$とおくと,
次の漸化式が成り立つ。
\begin{eqnarray*}
    &f(n, 0) = \dfrac{1}{s^n},\ f(n, n) = \dfrac{1}{s^n} \\
    &f(n, r) = \dfrac{1}{s} \{f(n-1, r-1) + f(n-1, r)\} \\
\end{eqnarray*}
このことを用いて,再帰的に確率の値を計算することができる。

\end{document}
