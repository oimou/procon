\documentclass{article}
\usepackage{parskip}
\usepackage{amsmath}
\begin{document}

点$(X, Y)$を$P$とおく。
移動の対称性により,$P$は$X \geq 0, Y \geq 0$の範囲にあるとしても一般性を失わない。
以下,この前提のもとで考える。

$P$に到達できる移動方法が存在するためには,$X$も$Y$も$D$で割り切れることが必要である。
このとき,ある非負整数$k, l$が存在し,次が成り立つ。
\begin{eqnarray*}
    X &=& kD \\
    Y &=& lD
\end{eqnarray*}
よって,原点から$P$までのマンハッタン距離は$X + Y = (k + l)D$と表すことができる。
このことと,$N$回で移動できるマンハッタン距離の最大値は$ND$であることから,
\begin{equation}
    \label{N_k_l}
    N \geq k + l
\end{equation}
が必要である。

ここで,
\begin{quote}
    $X$軸に平行に$+D, -D$だけ移動する回数をそれぞれ$a, b$ \\
    $Y$軸に平行に$+D, -D$だけ移動する回数をそれぞれ$c, d$
\end{quote}
とおく。

ちょうど$N$回で$P$に到達するような移動の仕方は,非負整数$k', l'$を用いて表すと,
\begin{quote}
    $X$軸の正方向に$k + k'$回,負方向に$k'$回だけ移動 \\
    $Y$軸の正方向に$l + l'$回,負方向に$l'$回だけ移動
\end{quote}
するような場合なので,
$$a = k + k',\ b = k',\ c = l + l',\ d = l'$$
である。このような移動の仕方を「ゴールできる移動の仕方」と呼ぶことにする。
ただし,$k', l'$は
$$N = a + b + c + d = (k + k') + k' + (l + l') + l'$$
を満たすので,
$$l' = \frac{1}{2} (N - k - 2k' - l)$$
が成り立ち,このような整数$l'$が存在するためには$N-(k+l)$が偶数であることが必要である。
この条件のもとで,$0 \leq k' \leq N, 0 \leq l' \leq N$から,$k'$の範囲は,
\begin{equation}
    \label{k_range}
    0 \leq k' \leq \frac{1}{2} (N - (k + l))
\end{equation}
であり,(\ref{N_k_l})の制約のもとでこれを満たす$k'$が存在する。

「ゴールできる移動の仕方」の個数は,$k'$を固定すると,
\[
    {}_N C_a \times {}_{N-a} C_b \times {}_{N-a-b} C_c = \cfrac{N!}{a!b!c!d!}
\]
である。
$k'$は(\ref{k_range})の範囲をくまなく動くので,「ゴールできる移動の仕方」の個数は,
\[
    \sum_{k' = 0}^{ \frac{1}{2} (N - (k + l)) } \cfrac{N!}{a!b!c!d!}
\]
である。
$P$までの移動の仕方は$4^N$通りであり,これらの起こり方は同様に確からしいから,求める確率は,
\[
    \cfrac{1}{4^N} \sum_{k' = 0}^{ \frac{1}{2} (N - (k + l)) } \cfrac{N!}{a!b!c!d!}
\]
である。

\end{document}
