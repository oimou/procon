\documentclass{article}
\usepackage{parskip}
\usepackage{amsmath}
\usepackage[dvipdfmx]{graphicx}
\usepackage{cleveref}

\begin{document}

整数列$\{L_n\}$の項数を$N (\geq 3)$とおく。
$\{L_n\}$は広義単調増加であるとしてよい。

$\{L_n\}$から第$i, j$項$(i < j)$を選んでそれぞれ$a = L_i, b = L_j$とおき,
\begin{equation}
    \label{c_upper_limit}
    a + b \leq L_n
\end{equation}
をみたす最小の$n$を$l_{a, b}$とおく。
ただし,$\{L_n\}$の最後の項が$a + b$より小さいとき,およびそのときのみ,
(\ref{c_upper_limit})を満たす最小の$n$は存在しないから,
そのときは$l_{a, b} = N + 1$とおく。

$\{L_n\}$の単調増加性により,
整数$k$が閉区間$[j + 1, l_{a, b} - 1]$に含まれるとき,およびそのときのみ,
$L_k$は次を満たす:
\begin{equation}
    \label{c_range}
    b \leq L_k < a + b
\end{equation}
ここで,$1 \leq a \leq b \leq c$のとき,
\begin{equation*}
    \begin{cases}
        a < b + c \\
        b < c + a \\
        c < a + b
    \end{cases}
    \Longleftrightarrow
    c < a + b
\end{equation*}
であるから,すべての$a, b$の組合せに対して(\ref{c_range})を満たす$L_k$の個数,すなわち
\begin{equation*}
    \sum_{i = 1}^{N - 2} \sum_{j = i + 1}^{N - 1}
    \max \{ (l_{a, b} - 1) - (j + 1) + 1, 0 \}
\end{equation*}
が求める値である。

\end{document}
