\documentclass{article}
\usepackage{parskip}
\usepackage{amsmath}
\usepackage[dvipdfmx]{graphicx}
\usepackage{cleveref}

\begin{document}

数列$\{L_n\}$の項数を$N (\geq 3)$とおく。
$\{L_n\}$は広義単調増加であるとしても一般性を失わない。

$\{L_n\}$から第$i, j$項$(i < j)$を選んでそれぞれ$a, b$とおき,
\begin{equation}
    \label{c_upper_limit}
    a + b \leq L_n
\end{equation}
をみたす最小の$n$を$l$とおく。
ただし,$\{L_n\}$のすべての項が$a + b$より小さいとき,およびそのときのみ,
(\ref{c_upper_limit})を満たす最小の$n$が存在しないから,
そのときは$l = N + 1$とおく。

$\{L_n\}$の単調増加性により,
整数$k$が閉区間$[j + 1, l - 1]$に含まれるとき,およびそのときのみ,
$L_k$は次を満たす:
\begin{equation}
    \label{c_range}
    b \leq L_k < a + b
\end{equation}
ここで,$1 \leq a \leq b \leq c$のとき,
\begin{equation}
    \label{a}
    \begin{cases}
        a < b + c \\
        b < c + a \\
        c < a + b
    \end{cases}
    \Longleftrightarrow
    c < a + b
\end{equation}
であるから,(\ref{c_range})を満たす$L_k$の個数が求める値である。

% これだと {1, 1, 1, 1} (1通り)を正しくカウントできない

\end{document}
