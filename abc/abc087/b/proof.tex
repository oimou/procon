\documentclass{article}
\usepackage[dvipdfmx]{graphicx}
\usepackage{parskip}
\usepackage{ascmac}
\usepackage{amsmath,amssymb}
\usepackage{cleveref}
    \crefname{proposition}{命題}{命題}
    \crefname{theorem}{定理}{定理}
    \crefname{lemma}{補題}{補題}
    \crefmultiformat{lemma}{補題~#2#1#3}{,~#2#1#3}{, #2#1#3}{,~#2#1#3}
\usepackage{autonum}
\usepackage{amsthm}
    \makeatletter
    \renewenvironment{proof}[1][\proofname]{\par
        \pushQED{\qed}
        \normalfont
        \topsep6\p@\@plus6\p@ \trivlist
        \item[\hskip\labelsep{\bfseries #1}\@addpunct{\bfseries}]\ignorespaces
    }{%
        \popQED\endtrivlist\@endpefalse
    }
    \renewcommand{\proofname}{証明.}
    \makeatother
\newtheorem{proposition}{命題}
\newtheorem{theorem}{定理}
\newtheorem{lemma}{補題}
\newcommand{\myparagraph}[1]{\paragraph{#1}\mbox{}\\}
\newcommand{\combination}[2]{{}_{#1} \mathrm{C}_{#2}}
\newcommand{\myfloor}[1]{\left\lfloor {#1} \right\rfloor}
\newcommand{\myceil}[1]{\left\lceil {#1} \right\rceil}

\begin{document}

選んだ500円玉,100円玉,50円玉の枚数をそれぞれ$a, b, c$とおくと,
\begin{alignat}{1}
    & 500a + 100b + 50c = X \\
    & 0 \leq a \leq A \\
    & 0 \leq b \leq B \\
    & 0 \leq c \leq C
\end{alignat}
であるから,$a$の範囲は
\begin{equation}
    0 \leq a \leq \min \left\{ \myfloor{\dfrac{X}{500}}, A \right\}
\end{equation}
である。
この範囲で$a$を固定すると,
\begin{equation}
    b = \dfrac{X}{100} - 5a - \dfrac{c}{2}
\end{equation}
となる。
$c$の範囲は$0 \leq c \leq C$であるから,
条件を満たす$b$の範囲は
\begin{equation}
    \max \left\{ \myceil{\dfrac{X}{100} - 5a - \dfrac{C}{2}}, 0 \right\}
    \leq b \leq
    \min \left\{ \myfloor{\dfrac{X}{100} - 5a}, B \right\}
\end{equation}
である。
したがって,これを満たす組$(a, b)$の個数が求める値である。

\end{document}
