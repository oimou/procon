\documentclass{article}
\usepackage[dvipdfmx]{graphicx}
\usepackage{parskip}
\usepackage{ascmac}
\usepackage{amsmath,amssymb}
\usepackage{cleveref}
    \crefname{proposition}{命題}{命題}
    \crefname{theorem}{定理}{定理}
    \crefname{lemma}{補題}{補題}
    \crefmultiformat{lemma}{補題~#2#1#3}{,~#2#1#3}{, #2#1#3}{,~#2#1#3}
\usepackage{autonum}
\usepackage{amsthm}
    \makeatletter
    \renewenvironment{proof}[1][\proofname]{\par
        \pushQED{\qed}
        \normalfont
        \topsep6\p@\@plus6\p@ \trivlist
        \item[\hskip\labelsep{\bfseries #1}\@addpunct{\bfseries}]\ignorespaces
    }{%
        \popQED\endtrivlist\@endpefalse
    }
    \renewcommand{\proofname}{証明.}
    \makeatother
\usepackage[math]{cellspace}
    \cellspacetoplimit 4pt
    \cellspacebottomlimit 4pt
\usepackage{array}
\newtheorem{proposition}{命題}
\newtheorem{theorem}{定理}
\newtheorem{lemma}{補題}
\newcommand{\myparagraph}[1]{\paragraph{#1}\mbox{}\\}
\newcommand{\combination}[2]{{}_{#1} \mathrm{C}_{#2}}

\begin{document}

給料の額を$X$とおくと,$X$は次の確率分布に従う確率変数である。

\begin{table}[hb]
    \centering
    \everymath{\displaystyle}
    \begin{tabular}{|c|| *{4}{ >{$}Sc <{$}} |c|}
        \hline
        X & 10000 & 20000 & \dots & N \times 10000 & 計 \\
        \hline
        P & \frac{1}{N} & \frac{1}{N} & \dots & \frac{1}{N} & 1 \\
        \hline
    \end{tabular}
\end{table}

したがって、求める期待値は
\begin{alignat}{1}
    E(X) &= \sum_{k = 1}^{N} \dfrac{k \times 10000}{N} \\
         &= \dfrac{10000}{N} \times \dfrac{1}{2} N (N + 1) \\
         &= 5000 (N + 1)
\end{alignat}
である。

\end{document}
