\documentclass{article}
\usepackage[dvipdfmx]{graphicx}
\usepackage{parskip}
\usepackage{ascmac}
\usepackage{amsmath,amssymb}
\usepackage{cleveref}
    \crefname{proposition}{命題}{命題}
    \crefname{theorem}{定理}{定理}
    \crefname{lemma}{補題}{補題}
    \crefmultiformat{lemma}{補題~#2#1#3}{,~#2#1#3}{, #2#1#3}{,~#2#1#3}
\usepackage{autonum}
\usepackage{amsthm}
    \makeatletter
    \renewenvironment{proof}[1][\proofname]{\par
        \pushQED{\qed}
        \normalfont
        \topsep6\p@\@plus6\p@ \trivlist
        \item[\hskip\labelsep{\bfseries #1}\@addpunct{\bfseries}]\ignorespaces
    }{%
        \popQED\endtrivlist\@endpefalse
    }
    \renewcommand{\proofname}{証明.}
    \makeatother
\newtheorem{proposition}{命題}
\newtheorem{theorem}{定理}
\newtheorem{lemma}{補題}
\newcommand{\myparagraph}[1]{\paragraph{#1}\mbox{}\\}
\newcommand{\combination}[2]{{}_{#1} \mathrm{C}_{#2}}

\begin{document}

\section{計算手法に関して}

$p$が素数のとき次が成り立つ。ただし、合同式の法は$p$であり、$x'$は$x$のモジュラ逆数である。
\begin{alignat}{1}
    \combination{n}{k}
    &= \dfrac{n}{1} \cdot \dfrac{n - 1}{2} \cdot \dots \cdot \dfrac{n - k + 1}{k} \\
    &\equiv
        \dfrac{n}{1} \cdot \dfrac{n - 1}{2} \cdot \dots \cdot \dfrac{n - k + 1}{k}
        \cdot (1 \cdot 1') (2 \cdot 2') \cdots (k \cdot k') \\
    &=
        \dfrac{n}{1} \cdot \dfrac{n - 1}{2} \cdot \dots \cdot \dfrac{n - k + 1}{k}
        \cdot (1 \cdot 2 \cdot \dots \cdot k)(1' \cdot 2' \cdot \dots \cdot k') \\
    &=
        n (n - 1) \cdots (n - k + 1)
        (1' \cdot 2' \cdot \dots \cdot k') \\
    &=
        n \cdot 1' \cdot (n - 1) \cdot 2' \cdot \dots \cdot (n - k + 1) \cdot k' \\
\end{alignat}

\end{document}
