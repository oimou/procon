\documentclass{article}
\usepackage{parskip}
\usepackage{amsmath}
\begin{document}

一般に,正整数$n$の素因数分解が素数列$\{p_k\}$と非負整数列$\{\alpha_k\}$を用いて
$$n = p_1^{\alpha_1} p_2^{\alpha_2} \cdots p_k^{\alpha_k}$$
と表されるとき,$n$の正の約数の個数は
$$(\alpha_1 + 1) (\alpha_2 + 1) \cdots (\alpha_k + 1)$$
である。
したがって,これが8であるようなものの個数を数えればよい。

\end{document}
