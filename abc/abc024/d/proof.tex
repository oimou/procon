\documentclass{article}
\usepackage{parskip}
\usepackage{amsmath}
\usepackage{ascmac}
\usepackage[dvipdfmx]{graphicx}
\usepackage{cleveref}
\usepackage{amssymb}

\newcommand{\myparagraph}[1]{\paragraph{#1}\mbox{}\\}
\newcommand{\combination}[2]{{}_{#1} \mathrm{C}_{#2}}

\begin{document}

$M = 1000000007$とおく。
$(r, c), (r, c + 1), (r + 1, c)$にたどり着く方法の個数を
それぞれ$A', B', C'$とおくと,$A', B', C'$を$M$で割った余りがそれぞれ$A, B, C$である。

$x \geq 1,\ y \geq 1$のときマス$(x, y)$にたどり着く方法の個数は$\combination{x + y}{y}$であるから,
$A', B', C'$は次のように表せる:
\begin{equation*}
    \begin{cases}
        A' = \combination{r + c}{r} = \combination{r + c}{c} \\
        B' = \combination{r + c + 1}{c + 1} \\
        C' = \combination{r + c + 1}{r + 1}
    \end{cases}
\end{equation*}

\myparagraph{[1] $A \geq 2$のとき}

$A \geq 2$ならば$A' \geq 2$なので,$r \geq 1,\ c \geq 1$である。

一般に,$n \geq 2,\ k \geq 1$のとき
\begin{equation*}
    k \times \combination{n}{k} = n \times \combination{n - 1}{k - 1}
\end{equation*}
が成り立つから,$A', B', C'$に関して次が成り立つ:
\begin{equation*}
    \begin{cases}
        (r + 1)\ C' = (r + c + 1)\ A' \\
        (c + 1)\ B' = (r + c + 1)\ A'
    \end{cases}
\end{equation*}
$1 \leq A' < B',\ 1 \leq A' < C'$に注意してこれを解くと,
\begin{equation*}
    \begin{cases}
        \vspace{10pt}
        r = \dfrac{B'C' - A'C'}{-B'C' + A'B' + A'C'} \\
        c = \dfrac{B'C' - A'B'}{-B'C' + A'B' + A'C'}
    \end{cases}
\end{equation*}
である。

% TODO



\section{逆元の存在証明}

% TODO



\section{フェルマーの小定理の証明}

整数$m\ (\geq 2)$に対して
\begin{equation*}
    d_m = \mathrm{gcd} (\combination{m}{1}, \combination{m}{2},\ \ldots\ , \combination{m}{m-1})
\end{equation*}
と定義する。

\begin{itembox}[l]{補題1}
    $m$が素数ならば$d_m = m$が成り立つ。
\end{itembox}

一般に,$m \geq 2,\ k \geq 1$のとき
\begin{equation*}
    k \times \combination{m}{k} = m \times \combination{m - 1}{k - 1}
\end{equation*}
が成り立つ。
このことと,$1 \leq k \leq m - 1$のとき$m$と$k$は互いに素であることから,
任意の$k\ (1 \leq k \leq m - 1)$に対して
$\combination{m}{k}$は$m$で割り切れる。

また,$\combination{m}{1} = m$で$m$は素数なので$d_m = 1$または$d_m = m$が必要である。

したがって,$d_m = m$である。

\begin{itembox}[l]{補題2}
    任意の自然数$k$に対して,$k^m - k$は$d_m$で割り切れる。
\end{itembox}

自然数$n$に関する条件「$n^m - n$は$d_m$で割り切れる」を$P(n)$とおく。

$n = 1$のとき,$n^m - n$すなわち0は$d_m\ (\geq 1)$で割り切れるから,$P(1)$が成り立つ。

ある$n = k\ (\geq 1)$に対して$P(k)$が成り立つと仮定すると
\begin{equation}
    \label{assumption}
    k^m - k \equiv 0 \mod{d_m}
\end{equation}
が成り立つが,このとき
\begin{eqnarray*}
    (k + 1)^m - (k + 1) &=& \sum_{j = 0}^{m} \combination{m}{j} k^j - (k + 1) \\
                        &=& \sum_{j = 1}^{m - 1} \combination{m}{j} k^j + k^m - k \\
                        &\equiv& k^m - k \mod{d_m} \\
                        &\equiv& 0 \mod{d_m} \hspace{15pt} (\ \because (\ref{assumption}) )
\end{eqnarray*}
なので,$P(k + 1)$も成り立つ。

以上より,補題2が成り立つ。

\begin{itembox}[l]{フェルマーの小定理}
    $p$が素数のとき,任意の自然数$k$に対して$k^{p - 1} \equiv 1 \mod{p}$が成り立つ。
\end{itembox}

補題1, 2より,$p$が素数のとき,$p$と互いに素な任意の自然数$k$に対して
\begin{equation*}
    k^p \equiv k \mod{p}
\end{equation*}
が成り立つ。$k$と$p$は互いに素であるから,両辺を$k$で割ることができ,
\begin{equation*}
    k^{p - 1} \equiv 1 \mod{p}
\end{equation*}
が成り立つ。

\end{document}
