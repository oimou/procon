\documentclass{article}
\usepackage[dvipdfmx]{graphicx}
\usepackage{parskip}
\usepackage{ascmac}
\usepackage{amsmath,amssymb}
\usepackage{cleveref}
    \crefname{proposition}{命題}{命題}
    \crefname{theorem}{定理}{定理}
    \crefname{lemma}{補題}{補題}
    \crefmultiformat{lemma}{補題~#2#1#3}{,~#2#1#3}{, #2#1#3}{,~#2#1#3}
\usepackage{autonum}
\usepackage{amsthm}
    \makeatletter
    \renewenvironment{proof}[1][\proofname]{\par
        \pushQED{\qed}
        \normalfont
        \topsep6\p@\@plus6\p@ \trivlist
        \item[\hskip\labelsep{\bfseries #1}\@addpunct{\bfseries}]\ignorespaces
    }{%
        \popQED\endtrivlist\@endpefalse
    }
    \renewcommand{\proofname}{証明.}
    \makeatother
\newtheorem{proposition}{命題}
\newtheorem{theorem}{定理}
\newtheorem{lemma}{補題}
\newcommand{\myparagraph}[1]{\paragraph{#1}\mbox{}\\}
\newcommand{\combination}[2]{{}_{#1} \mathrm{C}_{#2}}

\begin{document}

\section{本題の証明}

\begin{screen}
    \begin{proposition}
        \label{prop:main}
        求める$r, c$は
        \begin{alignat}{1}
            &(BC - AC)(-BC + AB + AC)^{M - 2}, \\
            &(BC - AB)(-BC + AB + AC)^{M - 2}
        \end{alignat}
        をそれぞれ$M$で割った余りである。
    \end{proposition}
\end{screen}

\begin{proof}
    $M = 10^9 + 7$とおく。
    $(r, c), (r, c + 1), (r + 1, c)$にたどり着く方法の個数を
    それぞれ$A', B', C'$とおくと,$A', B', C'$を$M$で割った余りがそれぞれ$A, B, C$である。

    $x \geq 0,\ y \geq 0,\ (x, y) \neq (0, 0)$のとき
    マス$(x, y)$にたどり着く方法の個数は$\combination{x + y}{y}$であるから,
    $A', B', C'$は次のように表せる:
    \begin{equation}
        \begin{cases}
            A' = \combination{r + c}{r} = \combination{r + c}{c} \\
            B' = \combination{r + c + 1}{c + 1} \\
            C' = \combination{r + c + 1}{r + 1}
        \end{cases}
    \end{equation}

    \myparagraph{[1] $r \geq 1$かつ$c \geq 1$のとき}

    一般に,$n \geq 2,\ k \geq 1$のとき
    \begin{equation}
        k \times \combination{n}{k} = n \times \combination{n - 1}{k - 1}
    \end{equation}
    が成り立つから,$A', B', C'$に関して次が成り立つ:
    \begin{equation}
        \begin{cases}
            (r + 1)\ C' = (r + c + 1)\ A' \\
            (c + 1)\ B' = (r + c + 1)\ A'
        \end{cases}
    \end{equation}
    $1 \leq A' < B',\ 1 \leq A' < C'$に注意してこれを解くと,
    \begin{equation}
        \begin{cases}
            \label{r_c}
            \vspace{10pt}
            r = \dfrac{B'C' - A'C'}{-B'C' + A'B' + A'C'} \\
            c = \dfrac{B'C' - A'B'}{-B'C' + A'B' + A'C'}
        \end{cases}
    \end{equation}
    であるから,$r, c$は一意的に定まる。

    題意により$r, c$は$M$より小さい非負整数なので,
    $r, c$は(\ref{r_c})の右辺を$M$で割った余りに一致する。

    ここで,
    \begin{alignat}{1}
        S &= B'C' - A'C' \\
        T &= B'C' - A'B' \\
        U &= -B'C' + A'B' + A'C' \\
        s &= BC - AC \\
        t &= BC - AB \\
        u &= -BC + AB + AC
    \end{alignat}
    とおく。

    $M$と$U$は互いに素であるとしてよい。 % なぜ???

    $M$は素数なので,\cref{lemma:inv:1}により
    法$M$に関する$U, u$の逆元$U', u'\ (0 \leq U' < M,\ 0 \leq u' < M)$が唯一つ存在する。
    $r, c$が整数ゆえに$S, T$がそれぞれ$U$の倍数であることに注意すれば,
    次が成り立つ:
    \begin{alignat}{2}
        r &\equiv SU' & &\mod{M} \hspace{15pt} (\because \mbox{\cref{lemma:inv:2}}) \\
          &\equiv su' & &\mod{M} \hspace{15pt} (\because \mbox{\cref{lemma:inv:3}}) \\
        c &\equiv TU' & &\mod{M} \hspace{15pt} (\because \mbox{\cref{lemma:inv:2}}) \\
          &\equiv tu' & &\mod{M} \hspace{15pt} (\because \mbox{\cref{lemma:inv:3}})
    \end{alignat}
    このことと\cref{theorem:flt}から,
    \begin{equation}
        u' \equiv u^{M - 2} \mod{M}
    \end{equation}
    なので,$r, c$は$su^{M - 2},\ tu^{M - 2}$をそれぞれ$M$で割った余りである。
    したがって,\cref{prop:main}が成り立つ。

    \myparagraph{[2] $r = 0$または$c = 0$のとき}

    $A' = 1,\ B' = c + 1,\ C' = r + 1$であるから,(\ref{r_c})が成り立つ。
    したがって,[1]と同様の議論により\cref{prop:main}が成り立つ。
\end{proof}



\section{逆元の性質に関する証明}

\begin{screen}
    \begin{lemma}
        \label{lemma:inv:1}
        $M$を素数,$x$を$M$と互いに素な整数とすると,
        法$M$に関する$x$の逆元$x'$が$0 \leq x' < M$の範囲に唯一つ存在する。
    \end{lemma}
\end{screen}

\begin{proof}
    $x^n$ ($n$は自然数; $1 \leq n \leq M$)を$M$で割った余りをそれぞれ$a_n$とおくと,
    \begin{alignat}{2}
        x   &\equiv a_1   & &\mod{M} \\
        x^2 &\equiv a_2 & &\mod{M} \\
            &\vdots& & \\
        x^M &\equiv a_M & &\mod{M}
    \end{alignat}
    となる。任意の整数を$M$で割った余りは0から$M - 1$までの$M$通りであるから,
    鳩の巣原理により,$a_1$から$a_M$のうちで少なくとも一組の重複がある。
    そのような組のひとつを$a_i,\ a_j\ (1 \leq i < j \leq M)$とおくと,
    \begin{equation}
        x^j - x^i \equiv 0 \mod{M}
    \end{equation}
    であるから,
    \begin{equation}
        x \cdot x^{j - i - 1} \equiv 1 \mod{M}
    \end{equation}
    が成り立つ。
    したがって,$x^{j - i - 1}$は法$M$に関する$x$の逆元のひとつであるから,逆元の存在が示された。

    ここで,$x', x''\ (1 \leq x' < M,\ 1 \leq x'' < M)$を$x$の逆元とすると,
    \begin{alignat}{1}
        x \cdot x'  &\equiv 1 \mod{M} \\
        x \cdot x'' &\equiv 1 \mod{M}
    \end{alignat}
    より,
    \begin{equation}
        x (x' - x'') \equiv 0 \mod{M}
    \end{equation}
    であるから,
    \begin{equation}
        x' \equiv x'' \mod{M}
    \end{equation}
    が成り立ち,$1 \leq x' < M,\ 1 \leq x'' < M$により$x' = x''$である。
    したがって,逆元の一意性が示された。
\end{proof}

\begin{screen}
    \begin{lemma}
        \label{lemma:inv:2}
        任意の整数$x, y$に対して,法$M$に関する$y$の逆元$y'$が存在するならば,
        \begin{equation}
            \frac{xy}{y} \equiv xyy' \mod{M}
        \end{equation}
        が成り立つ。
    \end{lemma}
\end{screen}

\begin{proof}
    与式の左辺は$x$と等しく,右辺は$M$を法として$x$と合同であるから,与式が成り立つ。
\end{proof}

\begin{screen}
    \begin{lemma}
        \label{lemma:inv:3}
        $M$を素数とする。任意の整数$x, y$の法$M$に関する逆元が存在しそれぞれ$x', y'$であるとき,
        \begin{equation}
            x \equiv y \mod{M}
            \hspace{15pt} \Longrightarrow \hspace{15pt}
            x' \equiv y' \mod{M}
        \end{equation}
        が成り立つ。
    \end{lemma}
\end{screen}

\begin{proof}
    逆元の性質により
    \begin{equation}
        xx' \equiv yy' \mod{M}
    \end{equation}
    が成り立つ。
    $M$が素数であることと,逆元が存在することから,$M$と$x$,$M$と$y$はそれぞれ互いに素である。
    よって,辺々を$x, y$で割って
    \begin{equation}
        x' \equiv y' \mod{M}
    \end{equation}
    が成り立つ。
\end{proof}



\section{フェルマーの小定理の証明}

整数$m\ (\geq 2)$に対して
\begin{equation}
    d_m = \mathrm{gcd} (\combination{m}{1}, \combination{m}{2},\ \ldots\ , \combination{m}{m-1})
\end{equation}
と定義する。

\begin{screen}
    \begin{lemma}
        \label{lemma:flt:1}
        $m$が素数ならば$d_m = m$が成り立つ。
    \end{lemma}
\end{screen}

\begin{proof}
    一般に,$m \geq 2,\ k \geq 1$のとき
    \begin{equation}
        k \times \combination{m}{k} = m \times \combination{m - 1}{k - 1}
    \end{equation}
    が成り立つ。
    このことと,$1 \leq k \leq m - 1$のとき$m$と$k$は互いに素であることから,
    任意の$k\ (1 \leq k \leq m - 1)$に対して
    $\combination{m}{k}$は$m$で割り切れる。

    また,$\combination{m}{1} = m$で$m$は素数なので$d_m = 1$または$d_m = m$が必要である。

    したがって,$d_m = m$である。
\end{proof}

\begin{screen}
    \begin{lemma}
        \label{lemma:flt:2}
        任意の自然数$k$に対して,$k^m - k$は$d_m$で割り切れる。
    \end{lemma}
\end{screen}

\begin{proof}
    自然数$n$に関する条件「$n^m - n$は$d_m$で割り切れる」を$P(n)$とおく。

    $n = 1$のとき,$n^m - n$すなわち0は$d_m\ (\geq 1)$で割り切れるから,$P(1)$が成り立つ。

    ある$n = k\ (\geq 1)$に対して$P(k)$が成り立つと仮定すると
    \begin{equation}
        \label{assumption}
        k^m - k \equiv 0 \mod{d_m}
    \end{equation}
    が成り立つが,このとき
    \begin{alignat}{2}
        (k + 1)^m - (k + 1) &= \sum_{j = 0}^{m} \combination{m}{j} k^j - (k + 1) & & \\
                            &= \sum_{j = 1}^{m - 1} \combination{m}{j} k^j + k^m - k & & \\
                            &\equiv k^m - k & &\mod{d_m} \\
                            &\equiv 0 & &\mod{d_m} \hspace{15pt} (\ \because (\ref{assumption}) )
    \end{alignat}
    なので,$P(k + 1)$も成り立つ。

    以上より,\cref{lemma:flt:2}が成り立つ。
\end{proof}

\begin{itembox}[l]{フェルマーの小定理}
    \begin{theorem}
        \label{theorem:flt}
        $p$が素数のとき,任意の自然数$k$に対して$k^{p - 1} \equiv 1 \mod{p}$が成り立つ。
    \end{theorem}
\end{itembox}

\begin{proof}
    \cref{lemma:flt:1,lemma:flt:2}より,$p$が素数のとき,$p$と互いに素な任意の自然数$k$に対して
    \begin{equation}
        k^p \equiv k \mod{p}
    \end{equation}
    が成り立つ。$k$と$p$は互いに素であるから,両辺を$k$で割ることができ,
    \begin{equation}
        k^{p - 1} \equiv 1 \mod{p}
    \end{equation}
    が成り立つ。
\end{proof}

\end{document}
