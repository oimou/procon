\documentclass{article}
\usepackage{parskip}
\usepackage{amsmath}
\usepackage[dvipdfmx]{graphicx}
\usepackage{cleveref}

\newcommand{\myparagraph}[1]{\paragraph{#1}\mbox{}\\}
\newcommand{\combination}[2]{{}_{#1} \mathrm{C}_{#2}}

\begin{document}

$M = 1000000007$とおく。
$(r, c), (r, c + 1), (r + 1, c)$にたどり着く方法の個数を
それぞれ$A', B', C'$とおくと,$A', B', C'$を$M$で割った余りがそれぞれ$A, B, C$である。

$x \geq 1,\ y \geq 1$のときマス$(x, y)$にたどり着く方法の個数は$\combination{x + y}{y}$であるから,
$A', B', C'$は次のように表せる:
\begin{equation*}
    \begin{cases}
        A' = \combination{r + c}{r} = \combination{r + c}{c} \\
        B' = \combination{r + c + 1}{c + 1} \\
        C' = \combination{r + c + 1}{r + 1}
    \end{cases}
\end{equation*}

\myparagraph{[1] $A \geq 2$のとき}

$A \geq 2$ならば$A' \geq 2$なので,$r \geq 1,\ c \geq 1$である。

一般に,$n \geq 2,\ k \geq 1$のとき
\begin{equation*}
    k \times \combination{n}{k} = n \times \combination{n - 1}{k - 1}
\end{equation*}
が成り立つから,$A', B', C'$に関して次が成り立つ:
\begin{equation*}
    \begin{cases}
        (r + 1)\ C' = (r + c + 1)\ A' \\
        (c + 1)\ B' = (r + c + 1)\ A'
    \end{cases}
\end{equation*}
$1 \leq A' < B',\ 1 \leq A' < C'$に注意してこれを解くと,
\begin{equation*}
    \begin{cases}
        \vspace{10pt}
        r = \dfrac{B'C' - A'C'}{-B'C' + A'B' + A'C'} \\
        c = \dfrac{B'C' - A'B'}{-B'C' + A'B' + A'C'}
    \end{cases}
\end{equation*}
である。

\end{document}
