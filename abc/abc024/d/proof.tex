\documentclass{article}
\usepackage{parskip}
\usepackage{amsmath}
\usepackage[dvipdfmx]{graphicx}
\usepackage{cleveref}

\newcommand{\myparagraph}[1]{\paragraph{#1}\mbox{}\\}
\newcommand{\combination}[2]{{}_{#1} \mathrm{C}_{#2}}

\begin{document}

$x \geq 1,\ y \geq 1$のときマス$(x, y)$の値は$\combination{x + y}{y}$であるから,
$A, B, C$はある非負整数$r, c$を用いて次のように表せる:
\begin{equation*}
    \begin{cases}
        A = \combination{r + c}{r} = \combination{r + c}{c} \\
        B = \combination{r + c + 1}{r} = \combination{r + c + 1}{c + 1} \\
        C = \combination{r + c + 1}{r + 1}
    \end{cases}
\end{equation*}

\myparagraph{[1] $A = 0$または$A = 1$のとき}

明らかに$r = B - 1,\ c = C - 1$が成り立つ。

\myparagraph{[2] $A \geq 2$のとき}

$A \geq 2$ならば$r \geq 1,\ c \geq 1$である。

一般に,$n \geq 2,\ k \geq 1$のとき
\begin{equation*}
    k \times \combination{n}{k} = n \times \combination{n - 1}{k - 1}
\end{equation*}
が成り立つから,$A, B, C$に関して次が成り立つ:
\begin{equation*}
    \begin{cases}
        (r + 1)\ C = (r + c + 1)\ A \\
        (c + 1)\ B = (r + c + 1)\ A
    \end{cases}
\end{equation*}
$\beta = \dfrac{B}{A},\ \gamma = \dfrac{C}{A}$とおき,
$1 \leq A < B,\ 1 \leq A < C$より$\beta > 1,\ \gamma > 1$であることに注意してこれを解くと,
\begin{equation*}
    \begin{cases}
        \vspace{10pt}
        r = \dfrac{(\beta  - 1) \gamma}{1 - (\beta - 1)(\gamma - 1)} \\
        c = \dfrac{(\gamma - 1) \beta }{1 - (\beta - 1)(\gamma - 1)}
    \end{cases}
\end{equation*}
である。

\end{document}
